%Appendeix2

%Not really sure what this pdf is, but I like it, and will post the link so I don't forget about it https://spie.org/samples/TT69.pdf
\section{Fluorescent Confocal Microscopy}
Confocal Microscopy is a technique capable of probing a sample with true 3-Dimensional optical resolution. The technique works by only allowing in focus light through a pinhole. By changing the plane of focus, one can traverse an object azimuthally or reconstruct a full three-dimensional image by compiling a ``stack'' of these images.  

%\begin{figure}[h!]
%	\centering
%	\floatbox[{\capbeside\thisfloatsetup{capbesideposition={left,top},capbesidewidth=4cm}}]{figure}[\FBwidth]
%	{\caption[Confocal Optics 1]{PLACE HOLDER. MAKE OWN VERSION OF DIAGRAM SO YOU DON'T HAVE TO CITE WHEREVER THIS CAME FROM}\label{fig:confocal_1}}
%	\includegraphics[width=0.4\linewidth]{confocal_stuff/confocal_orig_fi_1-1}
%\end{figure}

\begin{wrapfigure}{r}{.5\textwidth}
	
	\centering
	\vspace{-1.2em}
	\includegraphics[width=\linewidth]{confocal_stuff/confocal_orig_fi_1-1}
	\caption[Confocal Optics 1]{PLACE HOLDER.}
	\label{fig:confocal_1}
	
	
\end{wrapfigure}
	
Fluorescent Confocal Microscopy works by the same general principle, only with the added complexity of optically-excitable fluorophores. These fluorophores absorb light from a narrow range of frequencies and re-emmit fluorescent light at a different known and specific frequency, which can then be measured. This allows for a greater resolution to larger objects, as well as the ability to detect objects that would otherwise be invisible to the microscope.  

Fluorescent Confocal Microscopy works by illuminating the specimen with a laser. The laser light passes through a pinhole, then is reflected by the dichroic mirror and then focused by the objective lens on a small area of the specimen. The re-emitted light from the fluorophores has a longer wavelength than the laser, so it is then transmitted through mirror.  It then then passes through the pinhole where it is travels to the photo-detector.

\begin{figure}[h]
	\centering
	\includegraphics[width=.8\linewidth]{confocal_stuff/confocalpatent}
	\caption[Minsky Patent Diagram]{Also placeholder. I think my plan is to break this into chunks, though it almost deserves it's own appendix.}
	\label{fig:confocalpatent}
\end{figure}
\cite{patent:3013467} 

\subsection{Preparing a Fluorescent Bead Solution}
Make sure to also note that the solution should be wrapped to avoid light exposure and to keep them in the refrigerator. \todo[inline, color=yellow]{I think the actual borate buffer recipe might belong in an appendix}

\section{Image Analysis}
In the following section, we will step through the process of analyzing a single sphere. Recall that in order to determine the surface stress and adhesion energy, we must fit a line to a collection of spheres ranging in size.  

 \todo[inline, color=yellow]{Run through sphere011 for the Gelest 9:1 on glass that looked pretty}