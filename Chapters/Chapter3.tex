%Appendeix2

%Not really sure what this pdf is, but I like it, and will post the link so I don't forget about it https://spie.org/samples/TT69.pdf
\section{Fluorescent Confocal Microscopy}
Confocal Microscopy is a technique capable of probing a sample with true 3-Dimensional optical resolution. The technique works by only allowing in focus light through a pinhole. By changing the plane of focus, one can traverse an object azimuthally or reconstruct a full three-dimensional image by compiling a ``stack'' of these images.  

\begin{wrapfigure}{r}{.5\textwidth}
	
	\centering
	\vspace{-1.2em}
	\includegraphics[width=\linewidth]{confocal_stuff/confocal_orig_fi_1-1}
	\caption[Confocal Optics 1]{PLACE HOLDER.}
	\label{fig:confocal_1}
	
	
\end{wrapfigure}
	
Fluorescent Confocal Microscopy works by the same general principle, only with the added complexity of optically-excitable fluorophores. These fluorophores absorb light from a narrow range of frequencies and re-emmit fluorescent light at a different known and specific frequency, which can then be measured. This allows for a greater resolution to larger objects, as well as the ability to detect objects that would otherwise be invisible to the microscope.  

Fluorescent Confocal Microscopy works by illuminating the specimen with a laser. The laser light passes through a pinhole, then is reflected by the dichroic mirror and then focused by the objective lens on a small area of the specimen. The re-emitted light from the fluorophores has a longer wavelength than the laser, so it is then transmitted through mirror.  It then then passes through the pinhole where it is travels to the photo-detector.

\begin{figure}[h!]
	\centering
	\includegraphics[width=.3\linewidth]{confocal_stuff/confocalpatent}
	\caption[Minsky Patent Diagram]{Also placeholder. I think my plan is to break this into chunks, though it almost deserves it's own appendix.}
	\label{fig:confocalpatent}
\end{figure}
\cite{patent:3013467} 

\subsection{Preparing a Fluorescent Bead Solution}
Make sure to also note that the solution should be wrapped to avoid light exposure and to keep them in the refrigerator. \todo[inline, color=yellow]{I think the actual borate buffer recipe might belong in an appendix}

\section{Image Analysis}
In the following section, we will step through the process of analyzing a single sphere. Recall that in order to determine the surface stress and adhesion energy, we must fit a line to a collection of spheres ranging in size.  
\todo[inline,color=yellow]{Insert raw data file....currently it's in chapter 4. It should be in here instead}

\subsection{Particle Location}
To turn a raw image file (saved as a .ome.tif) into usable data, we use sophisticated MATLAB program to locate each fluorescent bead. This is plotted in \ref{fig:190215g91glasssphere011surface}. The flat orange plane of points is the surface, and the dark blue are the fluorescent beads crushed underneath the microsphere as it sinks into the substrate. It is easy to see the outline of the microsphere, and we can use this information to determine d and R for the sphere. 
\begin{figure}[h!]
	\centering
	\includegraphics[width=\linewidth]{Chapters/Figures/sphere011_ia/particle_located_normalized}
	\caption[Particle Located: Normalized-Axes]{}
	\label{fig:particlelocatednormalized}
\end{figure}
Figure \ref{fig:particlelocatedstretched} shows the same sphere but stretched in the vertical axis to see the fluorescent bead density towards the bottom of the sphere. 

\begin{figure}[h!]
	\centering
	\includegraphics[width=\linewidth]{Chapters/Figures/sphere011_ia/particle_located_stretched}
	\caption[Particle Located: Stretched-Axes]{Fluorescent bead location for the same sphere as Fig.  \ref{fig:particlelocatedstretched}, only with a stretched vertical axis to showcase the bead density under the bottom of the sphere.}
	\label{fig:particlelocatedstretched}
\end{figure}

\begin{figure}[h]
	\centering
	\includegraphics[width=\linewidth]{Chapters/Figures/sphere011_ia/particle_located_top_view}
	\caption[Particle Located: Top View]{}
	\label{fig:particlelocatedtopview}
\end{figure}
\subsection{Depth and Radius Determination}
After locating all the fluorescent beads and reconstructing our raw image in a useful format, we can then use this information to extract the depth and the radius of the sphere. To accomplish this, we first determine the center of out sphere from the top (x-y plane) and collapse the side profile of the \todo{clunky} sphere. Figure \ref{fig:sidecollapsed} depicts the x-z plane as a cross-section through center of the sphere. We only need half of the sphere's profile to fit a circle. This ensures we have a large field of view to properly level and determine the substrate's surface. Figure \ref{fig:sidecollapsedzoomed} provides a zoomed in view of Fig. \ref{fig:sidecollapsed} to better indicate the resolution to which we try to fit a circle. Note the abrupt cut off around 26 $\mu$m and how the substrate is lifted up above the zero-plane due to adhesion. Also notice the flatness on of the substrate between $ ~26-29 \mu $m. Not all sample have this flat ridge; many sphere lift the substrate to a point which the immediately slopes downward back to the zero-plane. This behavior depends slightly on the sphere size, but it is mainly a reflection of the amount of phase-separation induced. This is a property of the each silicone which depends on the environmental conditions.

It is important to be careful with the circle-fits in the case where the sphere is small enough to sink into the substrate beyond the great circle. In these instances, there the side profile is non-continuous, and we only fit the sphere to the bottom portion of the profile before the break.

\begin{figure}
	\centering
	\includegraphics[width=\linewidth]{Chapters/Figures/sphere011_ia/side_collapsed}
	\caption[Collapsed Side Profile]{The constructed image of the outlined sphere with normalized axes. The Axes are in microns.}
	\label{fig:sidecollapsed}
\end{figure}
\begin{figure}
	\centering
	\includegraphics[width=\linewidth]{Chapters/Figures/sphere011_ia/side_collapsed_zoomed}
	\caption[Collapsed Side Profile: Zoomed]{}
	\label{fig:sidecollapsedzoomed}
\end{figure}

\begin{figure}
	\centering
	\includegraphics[width=\linewidth]{Chapters/Figures/sphere011_ia/circle_fit}
	\caption[Circle Fit]{Notice how the surface plane is completely flat and centered at a depth of 0.}
	\label{fig:circlefit}
\end{figure}
\begin{figure}
	\centering
	\includegraphics[width=\linewidth]{Chapters/Figures/sphere011_ia/circle_fit_zoomed}
	\caption[Circle Fit Zoomed]{The orange dots are the data points being used to fit the circle. Notice how the sphere fits nicely in the indentation all the way up to the cusp, past the points being used for the fit.}
	\label{fig:circlefitzoomed}
\end{figure}
\begin{figure}
	\centering
	\includegraphics[width=\linewidth]{Chapters/Figures/sphere011_ia/single_d_vs_r}
	\caption[D vs R plot]{}
	\label{fig:singledvsr}
\end{figure}
