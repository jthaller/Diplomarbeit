%Appendeix2

%Not really sure what this pdf is, but I like it, and will post the link so I don't forget about it https://spie.org/samples/TT69.pdf

Confocal Microscopy is a technique capable of probing a sample with true 3-Dimensional optical resolution. The technique works by only allowing in focus light through a pinhole. By changing the plane of focus, one can traverse an object azimuthally or reconstruct a full three-dimensional image by compiling a ``stack'' of these images.  
\\ \\
	{\large{Insert Figure here} } 
\\ \\
Fluorescent Confocal Microscopy works by the same general principle, only with the added complexity of optically-excitable fluorophores. These fluorophores absorb light from a narrow range of frequencies and re-emmit fluorescent light at a different known and specific frequency, which can then be measured. This allows for a greater resolution to larger objects, as well as the ability to detect objects that would otherwise be invisible to the microscope.  

Fluorescent Confocal Microscopy works by illuminating the specimen with a laser. The laser light passes through a pinhole, then is reflected by the dichroic mirror and then focused by the objective lens on a small area of the specimen. The re-emitted light from the fluorophores has a longer wavelength than the laser, so it is then transmitted through mirror.  It then then passes through the pinhole where it is travels to the photo-detector.
\\ \\
{\large{Insert Figure here} \cite{patent:3013467} } 

\subsection{Preparing a Fluorescent Bead Solution}
I don't have my journal with me, but I will put the recipe for the borate buffer solution.....Make sure to also note that the solution should be wrapped to avoid light exposure and to keep them in the refrigerator. 
\\ \\