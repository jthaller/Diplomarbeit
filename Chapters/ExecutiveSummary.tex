% Here we have your executive summary.
This thesis describes a new method for measuring strain dependent surface stress in soft solids, as well as the corresponding measurements for variety of silicone. Over the past several years,recent attempts to measure surface stress in gels have returned a cornucopia of conflicting results, differing
significantly in similar materials. Professor Katharine Jensen suggested that this discrepancy is an artifact of the strain state of the gels, and recently made the first ever measurement of strain-dependent surface stress, $\Upsilon(\epsilon)$ in solids \cite{xu2017direct}. In this 2017 paper, Professor Katharine Jensen and her colleagues at ETH Zürich reported that the surface stress changed dramatically under applied strain. 
\begin{figure}[h!]
	\centering
	\textbf{Your title}\par\medskip
	\includegraphics[width=0.7\linewidth]{Chapters/Figures/2017natcomfig}
	\caption[Surface Stress vs. Strain in Silicone]{First direct measurement of strain dependent surface stress in solids \cite{xu2017direct}}
	\label{fig:2017natcomfig}
\end{figure}

\emph{Can I please have a scalable version of this figure? I think I got this from a screenshot...}  



Over the past year, we have helped develop a new and superior method
for measuring $\Upsilon(\epsilon)$ in soft solids. Unlike the technique published previously, our adhesion-based method can be applied to virtually any soft solid capable of being stretched. We have also build our own equibiaxial stretching apparatus, improving upon the earlier published design to produce nearly twice the strain in identical materials.

%This thesis describes the measurement and preliminary theory of how solid surface stress is dependent on strain in soft matter. Classical soft matter theory holds that surface tension for a gel should remain relatively constant under strain. This idea is based on the fact that gels are mostly liquid, and liquids do not exhibit strain-dependent surface tension.  However, a paper \cite{xu2017direct} published in 2017 made the first measurements demonstrating a significant link between surface tension and strain in solid matter. This paper was met with intrigue, but justified skepticism from the field.

%From Charlie:
%Your executive summary will give a detailed summary of your thesis, hitting the high points and perhaps including a figure or two.  This should have all of the important take-home messages; though details will of course be left for the thesis itself, here you should give enough detail for a reader to have a good idea of the content of the full document.  Importantly, this summary should be able to stand alone, separate from the rest of the document, so although you will be emphasizing the key results of your work, you will probably also want to include a sentence or two of introduction and context for the work you have done.


