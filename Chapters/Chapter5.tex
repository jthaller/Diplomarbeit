\section{Noise Minimization}
\todo[color = pink, inline]{Did I write about how I re-leveled the stretcher mount and I think this will solve the noise issue? I think that should probably go here.}

\section{New Materials}
The adhesion based technique for measuring $\Upsilon(\epsilon)$ should be applicable to virtually any soft solid capable of being stretched. Naturally, we would like to expand our measurements beyond silicone. Hydrogels, aerogels, gelatins, and commercial adhesives, to name a few, are materials of interest. 
\subsection{Neue Materialen: Gummibärli}
Measuring the strain dependent surface stress of a gummy bear presents challenges not present in PDMS silicone. For example, gummy bears come in a predefined size and shape. In order to spin coat them on our apparatus, we must first melt down the gummies. Gummy bears are composed primarily of gelatin and a glucose syrup.\footnote{American gummy bears use corn syrup. However, we are using Swiss  Gummibärli, which likely use a different variety}  When melted, they tend to re-solidify as brittle incompressible. To resolve this, \todo[color=pink,inline]{chat with KEJ about what Adam's done so far and if we solved this brittleness problem. I think just using a lower temperature would work though}
  
\section{Future Techniques}
Currently, we use equation \ref{THEeqn} to run a two parameter fit for the surface stress, $ \Upsilon $, and the adhesion energy, $ W $. Alternatively, it is possible to measure the adhesion energy separately. This technique, though not critically important to the experiment, is worth exploring. Measuring $ W $ separately and comparing the measured value to the calculated value via the two parameter fit would inform us about the accuracy of our fit.  

