The adhesion based technique for measuring $\Upsilon(\epsilon)$ should be applicable to virtually any soft solid capable of being stretched. Naturally, we would like to expand our measurements beyond silicone. Hydrogels, aerogels, and gelatins, to name a few, are materials of interest. Commercial adhesives are a subject of financial interest, with possible consequential technological developments.

\section{Neue Materialen: Gummibärli}
Measuring the strain dependent surface stress of a gummy bear presents challenges not present in PDMS silicone. For example, gummy bears come in a predefined size and shape. In order to spin coat them on our apparatus, we must first melt down the gummies. Gummy bears are composed primarily of gelatin and a glucose syrup. When melted...\todo[inline,color=pink]{I need to chat with Adam about this. This is what he's been doing. I'll send him an email.}

\section{Future Techniques}
Currently, we use equation \ref{THEeqn} to run a two parameter fit for the surface stress, $ \Upsilon $, and the adhesion energy, $ W $. Alternatively, it is possible to measure the adhesion energy separately. This technique, though not critically important to the experiment, is worth exploring. Measuring $ W $ separately and comparing the measured value to the calculated value via the two parameter fit would inform us about the accuracy of our fit.  