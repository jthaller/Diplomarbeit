\section{Noise Minimization}
As covered in the previous chapter, the limiting factor for our $ \Upsilon(\epsilon) $ and $ W(\epsilon) $ measurements is the noise present for all d vs. R data collected on stretchable substrates. We believe this noise is a result of the uneven mount that holds the stretching apparatus on the microscope's stage. We have corrected the mount's tilt by sanding and re-gluing the sides of the mount; however, if this does not correct the tilt, there are still options to consider. 

If the mount is still too uneven, we could put tiny piezoelectric step motors on three corners of the mount, allowing for precise leveling of the surface plane for each sphere during data collection. We believe this will not be necessary, however, and suspect the reduced tilt from its current state will be sufficient to level the surface plane in post via software.

Alternatively, we have been measuring the radius of the spheres by fitting a circle to the azimuthal collapse, or side profile (see Figure \ref{fig:circlefitzoomed}). Instead, we could fit a sphere to the three-dimensional indentation (such as Figure \ref{fig:particlelocatednormalized}). The advantage of fitting a sphere rather than a circle would be the increased number of points used for the fitting. We do not suspect the circle-fitting used for measuring $ R $ has been contributing to the noise observed. However, this spherical fitting technique may be useful in measuring r for very small spheres, which have been problematic for the circle fits.    


\section{New Materials}
The adhesion based technique for measuring $\Upsilon(\epsilon)$ should be applicable to virtually any soft solid capable of being stretched. Naturally, we would like to expand our measurements beyond silicone. Hydrogels, aerogels, gelatins, and commercial adhesives, to name a few, are materials of interest. 

In the lab, we have begun to explore hydrogels specifically, and have successfully spin-coated fluorescently marked glycerol-based gelatin gels onto glass coverslips appropriate for confocal microscopy. 


We are also pursuing measurements with commercially-manufactured hydrogels, namely Swiss Gummy Bears. Measuring the strain dependent surface stress of a gummy bear presents challenges not present in PDMS silicone. For example, gummy bears come in a predefined size and shape. In order to spin coat them on our apparatus, we must first melt down the gummies. Gummy bears are composed primarily of gelatin and a glucose syrup.\footnote{Unlike Gummib{\"a}rli, American gummy bears use corn syrup. We expect the landscape of gummy candies to provide a rich array of materials for future experiments.}
  
\section{Future Techniques}
Currently, we use equation \ref{THEeqn} to run a two parameter fit for the surface stress, $ \Upsilon $, and the adhesion energy, $ W $. Alternatively, it is possible to measure the adhesion energy separately. This technique, though not critically important to the experiment, is worth exploring. Measuring $ W $ separately and comparing the measured value to the calculated value via the two parameter fit would inform us about the accuracy of our fit for the zero-strain data, and possibly the strain-data if the adhesion energy per area is independent of strain, i.e. $ W = W(\epsilon) $.   

